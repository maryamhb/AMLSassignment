% ============================= Info ============================= %

% Applied Machine Learning Systems ELEC0132 Assignment
% Due 11:59pm, 07 Jan 2019

% 6 double-column single-spaced pages 
% Plus up to 4 pages of supplementary material 
% (methodology clarification, additional results, other tests, etc.)

% Include hidden link to code in public repo (Dropbox, Drive, etc.)

\iffalse
Assignment tasks:
	- Detection and removal of noisy images
	- Training, validation and testing subsets division
	- Train ML models to perform
		Binary tasks
			1. Emotion recognition (smile/!smile)
			2. Age identification (young/old)
			3. Glasses detection (with/without)
			4. Human detection (real/avatar)
		Multiclass tasks
			5. Hair colour recognition (ginger, blond, brown, grey, black, bald)
\fi

% =========================== Packages =========================== %
\documentclass[conference]{IEEEtran}
\usepackage{amsmath, graphicx, multicol, cite}
\usepackage{color}
\usepackage{listings}% code
%\usepackage{refcheck}
\lstset{escapeinside={<@}{@>}}

% ============================= Title ============================= %
\begin{document}

\title{Applied Machine Learning\\
 Systems ELEC0132 Assignment}
\author{\large Maryam Habibollahi\\ \textit{Department of Electronic and Electrical Engineering}\\ \textit{University College London}\\ zceemha@ucl.ac.uk}
\date{November 2018}
\maketitle

\setcounter{page}{1} \pagenumbering{arabic}

% ============================ Abstract =========================== %
\begin{center} \large \textbf{Abstract} \end{center}
\textit{Brief overview of the methodology/results presented.}\\

% ========================== Introduction ========================== %
\section{Introduction} \label{s-intro}

The problem statement.\\

Dataset description summarising data (content, size, format, etc.) and describing any \textit{data preprocessing} applied.\\

% ======================= Proposed algorithms ====================== %
\section{Proposed algorithms} \label{s-algorithms}

Algorithmic approach used to solve the problem.\\

Explain rationale behind choices, i.e. detail your \textit{reasons for selecting a particular model}.\\

% ========================= Implementation ======================== %
\section{Implementation} \label{s-implement}

Provide name and use of \textit{external libraries} and explain how \textit{model parameters} were selected.\\

Thorough discussion on the training convergence and stopping criterion (use learning curves graphs).\\

% ======================= Experimental results ====================== %
\section{Experimental result} \label{s-exp-res}

Describe and discuss results, compare to other approaches in literature or variations of ML solutions.\\

Include \textit{accuracy prediction scores on a separate test dataset, provided by the module organisers, but not used during training and validation}.\\

% ========================== Conclusion ========================== %
\section{Conclusion} \label{s-concl}

Summaries all findings and suggest direction for future improvement.\\

% ========================= Related work ========================== %
\section{Related Work} \label{s-rel-work}

Summarise latest reserach on the topic, discussing merits/disadvantages of diff approaches.\\

% ========================== References ========================== %

\bibliographystyle{IEEEtran}
\bibliography{/Users/MaryamH/Documents/BibTex/library}

\end{document}
